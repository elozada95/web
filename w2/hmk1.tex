\documentclass[11pt]{article}
\usepackage{amsmath,amssymb,amsthm}
\usepackage[utf8]{inputenc}
\usepackage{graphicx}
\usepackage[margin=1in]{geometry}
\usepackage{fancyhdr}
\setlength{\parindent}{0pt}
\setlength{\parskip}{5pt plus 1pt}
\setlength{\headheight}{13.6pt}
\newcommand\question[2]{\vspace{.25in}\hrule\textbf{#1: #2}\vspace{.5em}\hrule\vspace{.10in}}
\renewcommand\part[1]{\vspace{.10in}\textbf{(#1)}}
\newcommand\algorithm{\vspace{.10in}\textbf{Instrucciones: }}
\newcommand\correctness{\vspace{.10in}\textbf{Valor: }}
\pagestyle{fancyplain}
\lhead{\textbf{\NAME\ (\ADSOFTID)}}
\chead{\textbf{HW\HWNUM}}
\rhead{LabWeb2018, \today}
\begin{document}\raggedright

\newcommand\NAME{Adolfo Centeno}  
\newcommand\ADSOFTID{adsoft}     
\newcommand\HWNUM{1}              

\question{1}{Bootstrap} 

\part{a} \algorithm Analizar ejemplo de bootstrap de semana 1, 2 y hacer responsiva la pagina maps.html

\correctness 20 pts.

\question{2}{Marker}
\part{a} \algorithm Permitir cambiar el color del marcador, caracter y mensaje

\correctness 20 pts.

\question{3}{geocode}
\part{a} \algorithm Realizar diseño responsivo(formato libre), para mostrar los datos de direccion real

\correctness 20 pts.


\question{4}{openweather}
\part{a} \algorithm Realizar diseño responsivo(formato libre), para informacion de clima, obtener su propio Key de https://openweathermap.org

\correctness 20 pts.


\question{5}{Virtual private server}
\part{a} \algorithm Crear un repositorio, subirlo y clonarlo en el VPS, correr con Nodejs

\correctness 20 pts.

\end{document}
